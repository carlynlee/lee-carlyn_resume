

\documentclass{resume} % Use the custom resume.cls style

%\usepackage[left=0.75in,top=0.6in,right=0.75in,bottom=0.6in]{geometry} % Document margins
\usepackage[left=0.75in,top=0.6in,right=0.75in,bottom=0.3in]{geometry} % Document margins

\name{Carlyn Lee} % Your name
%\address{} % Your phone number and email
\renewcommand{\nameskip}{\smallskip}
\begin{document}


%----------------------------------------------------------------------------------------
%	TECHNICAL STRENGTHS SECTION
%----------------------------------------------------------------------------------------

%\begin{rSection}{Technical Strengths}

%\center{C/C++, C\#, Java, Python, Javascript, Matlab, Visual Basic, Excel, MEAN.io, OpenCV, OpenGL}

%\end{rSection}

%----------------------------------------------------------------------------------------
%	WORK EXPERIENCE SECTION
%----------------------------------------------------------------------------------------

\begin{rSection}{Experience}
\begin{rSubsection}{Jet Propulsion Laboratory, California Institute of Technology}{August 2012 - Present}{Applications Software Engineer}{Pasadena, CA}


\item As Europa Clipper Avionics I\&T test engineer: developed parameters testing of direct commands and uplinked files on WSTS and avionics testbed. Developed testbed scripts, software tools, VnV scripts to aid in automation of executing test procedures.

\item Mars2020 engineering operations telecom chair: assess health and status of telecom subsystems and relay links during tactical downlink shifts. Architected data delivery systems for DSN downlink streams and developed intelligent interfaces to reduce rover operator response-time to requirement of 20 minutes, aggregated data and event anomalies flagged by rules-engine and spacecraft housekeeping data.


\item Developed az-el terrain mask algorithm for massively parallel multiprocessor supercomputer to enable high fidelity communications link coverage mapping of potential lunar landing sites using Lunar Orbiter Laser Altimeter data. Parallelized Deep Space Optical Comm link channel coding simulations for ranges 0.25 AU - 1AU to enable signal and noise trade-space exploration in bit error rate regime of 10e-8. 


\item Communications support for collaborative multi-agent autonomy in maritime and subterranean environments. V\&V of software on networked Raspberry Pi’s and unmanned surface vehicles for collaborative operation of largest fleet of autonomous maritime vehicles. Radio mesh network trade studies in mining tunnel contributed to 1st place in DARPA’s Urban Circuit Subterranean Challenge. Ultra-wide band ROS integration to improve robot localization in GPS/comm deprived environments.

\item Telecom forecast prediction tools for various deep space missions, including full web stack development for SaaS application prototype. Implementation of network link models. Spacecraft, planetary, camera-matrix, and events analysis using C/C++. UX development for scheduling telecom links. Developed and integrated link performance \& SNR calculation and data volume modeling tools in python for use in Mars Relay Operations Service.

\item Implemented Link Complexity and Maintenance endpoint from events \& trends based on Sequence of Events files and modeled link complexity profile, enabled DSN sites to operate entire network during day shifts. 

\item Radio science operator for Cassini Spacecraft. Investigation of atmospheric losses for 32GHz radio communications recorded on Deep Space Network open \& closed loop receivers. 

\item Modeling of communications traffic flow for human exploration of Mars \& Moon. Python implementation of Markov model for estimating bandwidth requirements in Deep Space Network simulations.

\end{rSubsection}

%------------------------------------------------

\begin{rSubsection}{Spectral Imaging Laboratory}{November 2011 - Present}{Consultant}{Pasadena, CA}
\item Post-processing algorithm to correct for manufacturing inconsistencies in prototype of artificial compound eye.
\item Application of super resolution algorithms to ray-traced simulations of images captured with artificial compound eyes. Using Matlab and openCV, improved resolution of images degraded with noise models.
\item Modeling of visual acuity for multiple apertures on curved surface. Implementation of neural networks to improve angular resolution of a point light source. 
\end{rSubsection}



%------------------------------------------------

\begin{rSubsection}{California State University, Fullerton}{December 2009 - August 2012}{Research Assistant \& Intern}{Fullerton, CA}
\item Designed and implemented framework to improve run-time efficiency \& accuracy of cancer detection using eigen decomposition of DNA microarray data with large feature set.
\item Implemented framework to explore next generation sequencing alignment techniques for discovering binding sites in heat-shock proteins, integration of C/C++ self-organizing maps.
\item Developed scheduling tool for library resources using .NET framework. C\# student web application, VB.NET admin configuration tool. Database design \& implementation using SQL Server \& stored procedures.
\end{rSubsection}
\end{rSection}

%----------------------------------------------------------------------------------------
%	VOLUNTEER & PROFESSIONAL AFFILIATIONS SECTION
%----------------------------------------------------------------------------------------

\begin{rSection}{ Extracurricular, Volunteer \& Professional Affiliations}
{\bf 2014 - Present} Interplanetary Small Satellite Conference Committee Member\\
{\bf 2016 - 2022} Caltech Alpine Club Website Administrator. \\
{\bf 2019} Member of Duarte Ad Hoc Finance Advisory Committee, appointed by Duarte City Council. \\
{\bf 2010 - 2012} Vice-President of Association for Computing Machinery, CSU Fullerton. 
\end{rSection}



%----------------------------------------------------------------------------------------
%	Certifications SECTION
%----------------------------------------------------------------------------------------

%\begin{rSection}{Certifications}

%{\bf} Adult and Pediatric First Aid/CPR/AED, The American Red Cross (Certificate ID 00ISKSO).\\
%{\bf} Technician Class Amateur Radio Operator. \\
%{\bf} Private Pilot-ASEL +300hrs including instruction in PA28, C152, C172, CT210M.\\
%\end{rSection}
%----------------------------------------------------------------------------------------
%	EDUCATION SECTION
%----------------------------------------------------------------------------------------

\begin{rSection}{Education}
{\bf California State University, Fullerton}  \\ 
M.S. Computer Science \hfill {\em August 2012}\\
B.S. Computer Science, Minor in Mathematics  \hfill {\em July 2011}\\
\end{rSection}
%----------------------------------------------------------------------------------------
%	PUBLICATIONS SECTION
%----------------------------------------------------------------------------------------


\begin{rSection}{Publications}
\item Vander Hook, J. V., Seto, W., Nguyen, V., Hasnain, Z., Lee, C.-A., Gallagher,
L., Halpin-Chan, T., Varahamurthy, V., \& Angulo, M. (2022). Swarms of Pirates: Red Team Exercises Using
Autonomous High-Speed Maneuvering Surface Vessels. Field Robotics, 2, 872–909.

\item A. Agha et al.. NeBula: Quest for Robotic Autonomy in Challenging Environments; TEAM CoSTAR at the DARPA Subterranean Challenge. Journal of Field Robotics, 2021.

\item C. Lee, H. Xie, C.H. Lee, D. Lyakhov, and D. Michels. In Silico Methods for Space System Analysis: Optical Link Coding Performance and Lunar Terrain Masks. In AIAA ASCEND, Las Vegas, NV, 16-18 Nov. 2020.

\item D. Abraham, B. MacNeal, D. Heckman, Y. Chen, J. Wu, K. Tran, A. Kwok and C. Lee. Recommendations Emerging from an Analysis of NASA’s Deep Space Communications Capacity. In International Conference On Space Operations (SpaceOps 2018), Marseille, France, May 2018. 


\item J. Lad, M. Johnston, D. Tran, D. Brown, K. Roffo, and C. Lee. Complexity-Based Link Assignment for NASA’s Deep Space Network for Follow-the-Sun Operations. In International Conference On Space Operations (SpaceOps 2018), Marseille, France, May 2018. 

\item K. Pinover, M. Johnston, C. Lee. Optimizing SmallSat Scheduling for NASA’s Deep Space Network. In International Workshop on Planning and Scheduling for Space (IWPSS 2017). Pittsburgh, PA, June 2017. 


\item D. Morabito, D. Kahan, K. Oudrhiri, and C. Lee. Cassini Downlink Ka-Band Carrier Signal Analysis. The Interplanetary Network Progress Report, Volume 42-208, February 15, 2017. 


\item K. Cheung, D. Abraham, M. Sanchez-Net, K. Tran, C. Lee. Traffic modeling for Deep Space Network in the Human Exploration Era. In SpaceOps 2016 Conference, Daejeon, Korea, May 16-20, 2016.


\item M. Johnston, C. Lee, C. Lau, K. Cheung, M. Levesque, B. Carruth, A. Coffman, M. Wallace. Integrating space communication network capabilities via web portal technologies. In SpaceOps 2014 13th International Conference on Space Operations, Pasadena, California, May 5-9, 2014. 


\item C. Lee, C.H. Lee. Cancer Screening Using Multi-modal Differential Principal Orthogonal Decomposition. In 2013 13th International Conference on Computational Science and Its Applications, Ho Chi Minh City, Vietnam, June 24-27 2013.

\item C. Lee, C.H. Lee. Extended Principal Orthogonal Decomposition Method for Cancer Screening. International Journal of Bioscience, Biochemistry and Bioinformatics vol. 2, no. 2, pp. 136-141, 2012.

\item C. Lee. Rest architecture for link analysis tools portal. NASA Undergraduate Student Research Program (USRP), Pasadena, California, August 2011. 




\end{rSection}


%----------------------------------------------------------------------------------------
%	EXAMPLE SECTION
%----------------------------------------------------------------------------------------

%\begin{rSection}{Section Name}

%Section content\ldots

%\end{rSection}

%----------------------------------------------------------------------------------------

\end{document}
