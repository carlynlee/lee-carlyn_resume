

\documentclass{resume} % Use the custom resume.cls style

%\usepackage[left=0.75in,top=0.6in,right=0.75in,bottom=0.6in]{geometry} % Document margins
\usepackage[left=1in,top=1in,right=1in,bottom=1in]{geometry} % Document margins

\name{Carlyn Lee} % Your name
%\address{} % Your phone number and email
\renewcommand{\nameskip}{\smallskip}
\begin{document}


%----------------------------------------------------------------------------------------
%	TECHNICAL STRENGTHS SECTION
%----------------------------------------------------------------------------------------

%\begin{rSection}{Technical Strengths}

%\center{C/C++, C\#, Java, Python, Javascript, Matlab, Visual Basic, Excel, MEAN.io, OpenCV, OpenGL}

%\end{rSection}

%----------------------------------------------------------------------------------------
%	WORK EXPERIENCE SECTION
%----------------------------------------------------------------------------------------


\begin{rSection}{Summary}
Software Engineer with over ten years of experience designing and implementing scalable and reliable Deep Space Telecommunications software and data analysis pipelines. Expert in designing, developing, testing, and deploying applications to meet engineering and scientific user needs. Familiar with Python, Matlab, R, Java, C/C++. Strong communicator with experience collaborating cross-functionally to ensure data quality and analysis reproducibility. A passion for good software infrastructure, problem-solving, and contribute to research and scientific breakthroughs while building a culture of developing quality software and reproducible research. 

\end{rSection}

\begin{rSection}{Experience}
\begin{rSubsection}{Jet Propulsion Laboratory, California Institute of Technology}{August 2012 - Present}{Applications Software Engineer}{Pasadena, CA}


\item Designed and implemented efficient and horizontally-scalable spacecraft ground support software for processing and analyzing deep space telecommunications link data


\item Collaborated with cross-functional teams to develop new software features and conduct trade studies in support of research and development initiatives



\item Maintained high software reliability by adhering to sound engineering principles and thorough testing

 

\item Investigated and fixed issues to improve software quality and pioneered new initiatives to reduce issues



\item Developed infrastructure for cloud environments to support software development

\end{rSubsection}

%------------------------------------------------

\begin{rSubsection}{Spectral Imaging Laboratory}{November 2011 - Present}{Consultant}{Pasadena, CA}
\item Designed and implemented data processing pipelines and algorithms for prototypes of bio-inspired artificial compound eyes

\item Trained machine learning classifiers to improve the accuracy of visual acuity

\item Work closely with hardware prototypes towards product development

\end{rSubsection}

\begin{rSubsection}{California State University, Fullerton}{December 2009 - August 2012}{Research Assistant \& Intern}{Fullerton, CA}
\item Designed and implemented framework to improve run-time efficiency \& accuracy of cancer detection using eigen decomposition of DNA microarray data with large feature set.
\item Implemented framework to explore next generation sequencing alignment techniques for discovering binding sites in heat-shock proteins, integration of C/C++ self-organizing maps.
\end{rSubsection}

\end{rSection}

%----------------------------------------------------------------------------------------
%	VOLUNTEER & PROFESSIONAL AFFILIATIONS SECTION
%----------------------------------------------------------------------------------------

\begin{rSection}{ Skills }

\item Expertise with scientific computing tools: Tensorflow, NumPy, PyTorch, Jupyter Notebook.
\item Experienced with cross-functional collaborations, data quality, and analysis reproducibility. 
\item Familiarity with AWS services, continuous integration, best practices, and security.
\end{rSection}



%----------------------------------------------------------------------------------------
%	Certifications SECTION
%----------------------------------------------------------------------------------------

%\begin{rSection}{Certifications}

%{\bf} Adult and Pediatric First Aid/CPR/AED, The American Red Cross (Certificate ID 00ISKSO).\\
%{\bf} Technician Class Amateur Radio Operator. \\
%{\bf} Private Pilot-ASEL +300hrs including instruction in PA28, C152, C172, CT210M.\\
%\end{rSection}
%----------------------------------------------------------------------------------------
%	EDUCATION SECTION
%----------------------------------------------------------------------------------------

\begin{rSection}{Education}
{\bf California State University, Fullerton}  \\ 
M.S. Computer Science \hfill {\em August 2012}\\
B.S. Computer Science, Minor in Mathematics  \hfill {\em July 2011}\\
\end{rSection}
%----------------------------------------------------------------------------------------
%	PUBLICATIONS SECTION
%----------------------------------------------------------------------------------------


\begin{rSection}{Publications}
\item Vander Hook, J. V., Seto, W., Nguyen, V., Hasnain, Z., Lee, C.-A., Gallagher,
L., Halpin-Chan, T., Varahamurthy, V., \& Angulo, M. (2022). Swarms of Pirates: Red Team Exercises Using
Autonomous High-Speed Maneuvering Surface Vessels. Field Robotics, 2, 872–909.

\item A. Agha et al.. NeBula: Quest for Robotic Autonomy in Challenging Environments; TEAM CoSTAR at the DARPA Subterranean Challenge. Journal of Field Robotics, 2021.

\item C. Lee, H. Xie, C.H. Lee, D. Lyakhov, and D. Michels. In Silico Methods for Space System Analysis: Optical Link Coding Performance and Lunar Terrain Masks. In AIAA ASCEND, Las Vegas, NV, 16-18 Nov. 2020.

\item D. Abraham, B. MacNeal, D. Heckman, Y. Chen, J. Wu, K. Tran, A. Kwok and C. Lee. Recommendations Emerging from an Analysis of NASA’s Deep Space Communications Capacity. In International Conference On Space Operations (SpaceOps 2018), Marseille, France, May 2018. 


\item J. Lad, M. Johnston, D. Tran, D. Brown, K. Roffo, and C. Lee. Complexity-Based Link Assignment for NASA’s Deep Space Network for Follow-the-Sun Operations. In International Conference On Space Operations (SpaceOps 2018), Marseille, France, May 2018. 

\item K. Pinover, M. Johnston, C. Lee. Optimizing SmallSat Scheduling for NASA’s Deep Space Network. In International Workshop on Planning and Scheduling for Space (IWPSS 2017). Pittsburgh, PA, June 2017. 


\item D. Morabito, D. Kahan, K. Oudrhiri, and C. Lee. Cassini Downlink Ka-Band Carrier Signal Analysis. The Interplanetary Network Progress Report, Volume 42-208, February 15, 2017. 


\item K. Cheung, D. Abraham, M. Sanchez-Net, K. Tran, C. Lee. Traffic modeling for Deep Space Network in the Human Exploration Era. In SpaceOps 2016 Conference, Daejeon, Korea, May 16-20, 2016.


\item M. Johnston, C. Lee, C. Lau, K. Cheung, M. Levesque, B. Carruth, A. Coffman, M. Wallace. Integrating space communication network capabilities via web portal technologies. In SpaceOps 2014 13th International Conference on Space Operations, Pasadena, California, May 5-9, 2014. 


\item C. Lee, C.H. Lee. Cancer Screening Using Multi-modal Differential Principal Orthogonal Decomposition. In 2013 13th International Conference on Computational Science and Its Applications, Ho Chi Minh City, Vietnam, June 24-27 2013.

\item C. Lee, C.H. Lee. Extended Principal Orthogonal Decomposition Method for Cancer Screening. International Journal of Bioscience, Biochemistry and Bioinformatics vol. 2, no. 2, pp. 136-141, 2012.

\item C. Lee. Rest architecture for link analysis tools portal. NASA Undergraduate Student Research Program (USRP), Pasadena, California, August 2011. 




\end{rSection}


%----------------------------------------------------------------------------------------
%	EXAMPLE SECTION
%----------------------------------------------------------------------------------------

%\begin{rSection}{Section Name}

%Section content\ldots

%\end{rSection}

%----------------------------------------------------------------------------------------

\end{document}
