

\documentclass{resume} % Use the custom resume.cls style

\usepackage[left=1.0in,top=1.0in,right=1.0in,bottom=1.0in]{geometry} % Document margins

\name{Carlyn Lee} %  name
\address{626-419-6597 carlyn.lee@gmail.com} %  phone number and email
\renewcommand{\nameskip}{\smallskip}
\begin{document}


%----------------------------------------------------------------------------------------
%	TECHNICAL STRENGTHS SECTION
%----------------------------------------------------------------------------------------

\begin{rSection}{}
Accomplished Software Engineer specializing in architecting scalable solutions, pioneering system reproducibility strategies, and innovating advanced visualization tools for high-performance computing applications. Experienced in collaboration with interdisciplinary teams, designing complex algorithms, and optimizing computing systems with a variety of tools: 
\it{C/C++, Python, Java, MATLAB, Visual Basic}



\end{rSection}

%----------------------------------------------------------------------------------------
%	WORK EXPERIENCE SECTION
%----------------------------------------------------------------------------------------

\begin{rSection}{Experience}
\begin{rSubsection}{NASA Jet Propulsion Laboratory, California Institute of Technology}{August 2012 - November 2024}{Applications Software Engineer}{Pasadena, CA}


\item {\bf{Verification \& Validation:}} 
Personally responsible for developing automated tools and test procedures for avionics flight software, and executing tests on flight hardware. 


\item {\bf{Mission Operations:}}
As the Perseverance Rover telecom chair, responsible for the health and status of rover communications using data downlinked from Mars. Personally developed tools to reduce rover operation response-time from several hours to 20 minutes, exceeding requirements. 


\item {\bf{Supercomputing:}} 
Developed Lunar terrain mask algorithms for Shaheen II supercomputer, enabling 10m-resolution link coverage maps for lunar landing sites. Parallelized simulations for Deep Space Optical Communications, achieving bit error rate of 10e-8, a 1000x improvement over previous state-of-the-art, and enabling mission performance for VIPER. 
 

\item {\bf{Fieldwork:}} 
Responsible for reliable multi-agent data transfer processes occurring at the transport layer. Conducted field testing off autonomous systems in maritime and subterranean environments, contributing to 1st place in DARPA’s Urban Circuit Challenge and the deployment of the largest ever fleet of autonomous maritime vehicles. 


\item {\bf{Tools Development:}}
Built deep space mission development telecom simulation and modeling tools. Personally responsible for developing automation enabling three Deep Space Network ground antenna sites to manage the entire network during day shifts instead of operating 24\/7.


\end{rSubsection}

%------------------------------------------------

%\begin{rSubsection}{Spectral Imaging Laboratory}{November 2011 - Present}{Consultant}{Pasadena, CA}
%\item Post-processing algorithm to correct for manufacturing inconsistencies in prototype of artificial compound eye.
%\item Application of super resolution algorithms to ray-traced simulations of images captured with artificial compound eyes. Using Matlab and openCV, improved resolution of images degraded with noise models.
%\item Modeling of visual acuity for multiple apertures on curved surface. Implementation of neural networks to improve angular resolution of a point light source. 
%\end{rSubsection}



%------------------------------------------------

%\begin{rSubsection}{California State University, Fullerton}{December 2009 - August 2012}{Research Assistant \& Intern}{Fullerton, CA}
%\item Designed and implemented framework to improve run-time efficiency \& accuracy of cancer detection using eigen decomposition of DNA microarray data with large feature set.
%\item Implemented framework to explore next generation sequencing alignment techniques for discovering binding sites in heat-shock proteins, integration of C/C++ self-organizing maps.
%\item Developed scheduling tool for library resources using .NET framework. C\# student web application, VB.NET admin configuration tool. Database design \& implementation using SQL Server \& stored procedures.
%\end{rSubsection}
%\end{rSection}

%----------------------------------------------------------------------------------------
%	VOLUNTEER & PROFESSIONAL AFFILIATIONS SECTION
%----------------------------------------------------------------------------------------

\begin{rSection}{ Extracurricular, Volunteer \& Professional Affiliations}
{\bf 2019 - Present} Supporting the Global Network Advancement (GNA-G) Data Intensive Sciences working group \\
{\bf 2018 - Present} As a SoCal Linux Expo Volunteer helped set up infrastructure for on-site expo network and A\/V live stream recording for presentations\\
{\bf 2014 - Present} Interplanetary Small Satellite Conference Committee Member\\
{\bf 2016 - 2022} Caltech Alpine Club Website Administrator. \\
{\bf 2019} Member of Duarte Ad Hoc Finance Advisory Committee, appointed by Duarte City Council. \\
{\bf 2010 - 2012} Vice-President of Association for Computing Machinery, CSU Fullerton. 
\end{rSection}



%----------------------------------------------------------------------------------------
%	Certifications SECTION
%----------------------------------------------------------------------------------------

%\begin{rSection}{Certifications}

%{\bf} Adult and Pediatric First Aid/CPR/AED, The American Red Cross (Certificate ID 00ISKSO).\\
%{\bf} Technician Class Amateur Radio Operator. \\
%{\bf} Private Pilot-ASEL +300hrs including instruction in PA28, C152, C172, CT210M.\\
%\end{rSection}
%----------------------------------------------------------------------------------------
%	EDUCATION SECTION
%----------------------------------------------------------------------------------------

\begin{rSection}{Education}
{\bf California State University, Fullerton}  \\ 
M.S. Computer Science \hfill {\em August 2012}\\
B.S. Computer Science, Minor in Mathematics  \hfill {\em July 2011}\\
\end{rSection}
%----------------------------------------------------------------------------------------
%	PUBLICATIONS SECTION
%----------------------------------------------------------------------------------------


%\begin{rSection}{Publications}
%\item Vander Hook, J. V., Seto, W., Nguyen, V., Hasnain, Z., Lee, C.-A., Gallagher,
%L., Halpin-Chan, T., Varahamurthy, V., \& Angulo, M. (2022). Swarms of Pirates: Red Team Exercises Using
%Autonomous High-Speed Maneuvering Surface Vessels. Field Robotics, 2, 872–909.

%\item C. Lee, M. Shaikh, C.H. Lee and D. Michels. Lunar Terrain Coverage Analysis Data Delivery Workflow, AIAA 2021-4039. ASCEND 2021. November 2021.

%\item A. Agha et al.. NeBula: Quest for Robotic Autonomy in Challenging Environments; TEAM CoSTAR at the DARPA Subterranean Challenge. Journal of Field Robotics, 2021.

%\item C. Lee, H. Xie, C.H. Lee, D. Lyakhov, and D. Michels. In Silico Methods for Space System Analysis: Optical Link Coding Performance and Lunar Terrain Masks. In AIAA ASCEND, Las Vegas, NV, 16-18 Nov. 2020.

%\item D. Abraham, B. MacNeal, D. Heckman, Y. Chen, J. Wu, K. Tran, A. Kwok and C. Lee. Recommendations Emerging from an Analysis of NASA’s Deep Space Communications Capacity. In International Conference On Space Operations (SpaceOps 2018), Marseille, France, May 2018. 


%\item J. Lad, M. Johnston, D. Tran, D. Brown, K. Roffo, and C. Lee. Complexity-Based Link Assignment for NASA’s Deep Space Network for Follow-the-Sun Operations. In International Conference On Space Operations (SpaceOps 2018), Marseille, France, May 2018. 

%\item K. Pinover, M. Johnston, C. Lee. Optimizing SmallSat Scheduling for NASA’s Deep Space Network. In International Workshop on Planning and Scheduling for Space (IWPSS 2017). Pittsburgh, PA, June 2017. 


%\item D. Morabito, D. Kahan, K. Oudrhiri, and C. Lee. Cassini Downlink Ka-Band Carrier Signal Analysis. The Interplanetary Network Progress Report, Volume 42-208, February 15, 2017. 


%\item K. Cheung, D. Abraham, M. Sanchez-Net, K. Tran, C. Lee. Traffic modeling for Deep Space Network in the Human Exploration Era. In SpaceOps 2016 Conference, Daejeon, Korea, May 16-20, 2016.


%\item M. Johnston, C. Lee, C. Lau, K. Cheung, M. Levesque, B. Carruth, A. Coffman, M. Wallace. Integrating space communication network capabilities via web portal technologies. In SpaceOps 2014 13th International Conference on Space Operations, Pasadena, California, May 5-9, 2014. 


%\item C. Lee, C.H. Lee. Cancer Screening Using Multi-modal Differential Principal Orthogonal Decomposition. In 2013 13th International Conference on Computational Science and Its Applications, Ho Chi Minh City, Vietnam, June 24-27 2013.

%\item C. Lee, C.H. Lee. Extended Principal Orthogonal Decomposition Method for Cancer Screening. International Journal of Bioscience, Biochemistry and Bioinformatics vol. 2, no. 2, pp. 136-141, 2012.

%\item C. Lee. Rest architecture for link analysis tools portal. NASA Undergraduate Student Research Program (USRP), Pasadena, California, August 2011. 




%\end{rSection}


%----------------------------------------------------------------------------------------
%	EXAMPLE SECTION
%----------------------------------------------------------------------------------------

%\begin{rSection}{Section Name}

%Section content\ldots

%\end{rSection}

%----------------------------------------------------------------------------------------

\end{document}
