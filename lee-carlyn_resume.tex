%%%%%%%%%%%%%%%%%%%%%%%%%%%%%%%%%%%%%%%%%
% Medium Length Professional CV
% LaTeX Template
% Version 2.0 (8/5/13)
%
% This template has been downloaded from:
% http://www.LaTeXTemplates.com
%
% Original author:
% Trey Hunner (http://www.treyhunner.com/)
%
% Important note:
% This template requires the resume.cls file to be in the same directory as the
% .tex file. The resume.cls file provides the resume style used for structuring the
% document.
%
%%%%%%%%%%%%%%%%%%%%%%%%%%%%%%%%%%%%%%%%%

%----------------------------------------------------------------------------------------
%	PACKAGES AND OTHER DOCUMENT CONFIGURATIONS
%----------------------------------------------------------------------------------------

\documentclass{resume} % Use the custom resume.cls style

%\usepackage[left=0.75in,top=0.6in,right=0.75in,bottom=0.6in]{geometry} % Document margins
\usepackage[left=0.75in,top=0.6in,right=0.75in,bottom=0.3in]{geometry} % Document margins

\name{Carlyn Lee} % Your name
%\address{} % Your phone number and email
\begin{document}


%----------------------------------------------------------------------------------------
%	TECHNICAL STRENGTHS SECTION
%----------------------------------------------------------------------------------------

%\begin{rSection}{Technical Strengths}

%\center{C/C++, C\#, Java, Python, Javascript, Matlab, Visual Basic, Excel, MEAN.io, OpenCV, OpenGL}

%\end{rSection}

%----------------------------------------------------------------------------------------
%	WORK EXPERIENCE SECTION
%----------------------------------------------------------------------------------------

\begin{rSection}{Experience}
\begin{rSubsection}{Jet Propulsion Laboratory, California Institute of Technology}{August 2012 - Present}{Applications Software Engineer}{Pasadena, CA}
\item Code audit and testing of multi-agent maritime autonomy software for unmanned surface vehicles. Investigated race conditions in asynchronous engine operations from analysis of CAN messages.
\item Trade studies and implementation of radio mesh network in mining tunnel. Enabled collaborative autonomous agent capabilities and improved localization in tunnel environment for DARPA's Subterranean Challenge.
\item Architected data systems for COTS dashboards to explore aggregated data and event anomalies flagged by rules-engine, Event Verification Records \& Engineering Housekeeping, and Accountability data. e.g. Tableau \& Web Data Connectors, Kibana.
\item Implemented framework for Link Complexity and Maintenance software to query events or trends from Sequence of Events files and estimate a link complexity profile.
\item Telecom forecast prediction tools for various deep space missions, including full web stack development for SaaS application prototype. Implementation of network link models. Spacecraft, planetary, camera-matrix, and events analysis using C/C++. UX development for scheduling telecom links, e.g. Liferay portlet, Drupal, D3, WebGL. 
\item Modeling of communications traffic flow for human exploration of Mars \& Moon. Python implementation of Markov model for estimating bandwidth requirements in Deep Space Network simulations. 
\item Radio science operator for Cassini Spacecraft. Investigation of atmospheric losses for 32GHz radio communications recorded on Deep Space Network open \& closed loop receivers. Prototyped data warehouse for quick visualization of radio science data from Cassini Spacecraft during 2004-2015. 
\end{rSubsection}

%------------------------------------------------

\begin{rSubsection}{Spectral Imaging Laboratory}{November 2011 - Present}{Consultant}{Pasadena, CA}
\item Post-processing algorithm to correct for manufacturing inconsistencies in prototype of artificial compound eye.
\item Application of super resolution algorithms to ray-traced simulations of images captured with artificial compound eyes. Using Matlab and openCV, improved resolution of images degraded with noise models.
\item Modeling of visual acuity for multiple apertures on curved surface. Implementation of neural networks to improve angular resolution of a point light source. 
\end{rSubsection}



%------------------------------------------------

\begin{rSubsection}{California State University, Fullerton}{December 2009 - August 2012}{Research Assistant \& Intern}{Fullerton, CA}
\item Designed and implemented framework to improve run-time efficiency \& accuracy of cancer detection using eigen decomposition of DNA microarray data with large feature set.
\item Implemented framework to explore next generation sequencing alignment techniques for discovering binding sites in heat-shock proteins, integration of C/C++ self-organizing maps.
\item Delivered scheduling tool for library resources using .NET framework. C\# student web application, VB.NET admin configuration tool. Database design \& implementation using SQL Server \& stored procedures.
\end{rSubsection}
\end{rSection}

%----------------------------------------------------------------------------------------
%	VOLUNTEER & PROFESSIONAL AFFILIATIONS SECTION
%----------------------------------------------------------------------------------------

\begin{rSection}{ Extracurricular, Volunteer \& Professional Affiliations}
{\bf} Interplanetary Small Satellite Conference Committee; Caltech Alpine Club Website Administrator; Private Pilot-ASEL +300hrs including instruction in PA28, C152, C172, CT210M; Technician Class Amateur Radio Operator \\
{\bf 2019} Member of Duarte Ad Hoc Finance Advisory Committee, appointed by Duarte City Council to review the City's existing revenues, expenditures and future financial forecast, discuss possible cost containment measures and revenue enhancements. \\
{\bf 2010 - 2012} Vice-President of Association for Computing Machinery, CSU Fullerton. 
\end{rSection}



%----------------------------------------------------------------------------------------
%	AWARDS AND HONORS SECTION
%----------------------------------------------------------------------------------------

\begin{rSection}{Awards \& Honors}
{\bf 2015}  3rd place Topcoder Open Finals API Hackathon. \\
{\bf 2013} 1st place Biotech Track, 15th Annual IEEE Biomedical Engineering Biotech Contest. \\
{\bf 2012} Anita Borg Scholarship, CSUPERB Travel Grant, Orange County Outstanding Engineering Student Award. \\
\end{rSection}

%----------------------------------------------------------------------------------------
%	EDUCATION SECTION
%----------------------------------------------------------------------------------------

\begin{rSection}{Education}
{\bf California State University, Fullerton}  \\ 
M.S. Computer Science \hfill {\em August 2012}\\
B.S. Computer Science, Minor in Mathematics  \hfill {\em July 2011}\\
\end{rSection}
%----------------------------------------------------------------------------------------
%	PUBLICATIONS SECTION
%----------------------------------------------------------------------------------------

\begin{rSection}{Publications}
\item D. Abraham, B. MacNeal, D. Heckman, Y. Chen, J. Wu, K. Tran, A. Kwok and C. Lee. Recommendations Emerging from an Analysis of NASA’s Deep Space Communications Capacity. In International Conference On Space Operations (SpaceOps 2018), Marseille, France, May 2018. 


\item J. Lad, M. Johnston, D. Tran, D. Brown, K. Roffo, C. Lee. Complexity-Based Link Assignment for NASA’s Deep Space Network for Follow-the-Sun Operations. In International Conference On Space Operations (SpaceOps 2018), Marseille, France, May 2018. 

\item K. Pinover, M. Johnston, C. Lee. Optimizing SmallSat Scheduling for NASA’s Deep Space Network. In International Workshop on Planning and Scheduling for Space (IWPSS 2017). Pittsburgh, PA, June 2017. 


\item D. Morabito, D. Kahan, K. Oudrhiri, and C. Lee. Cassini Downlink Ka-Band Carrier Signal Analysis. The Interplanetary Network Progress Report, Volume 42-208, February 15, 2017. 


\item K. Cheung, D. Abraham, M. Sanchez-Net, K. Tran, C. Lee. Traffic modeling for Deep Space Network in the Human Exploration Era. In SpaceOps 2016 Conference, Daejeon, Korea, May 16-20, 2016.


\item M. Johnston, C. Lee, C. Lau, K. Cheung, M. Levesque, B. Carruth, A. Coffman, M. Wallace. Integrating space communication network capabilities via web portal technologies. In SpaceOps 2014 13th International Conference on Space Operations, Pasadena, California, May 5-9, 2014. 


\item C. Lee, C.H. Lee. Cancer Screening Using Multi-modal Differential Principal Orthogonal Decomposition. In 2013 13th International Conference on Computational Science and Its Applications, Ho Chi Minh City, Vietnam, June 24-27 2013.


\item C. Lee. Rest architecture for link analysis tools portal. NASA Undergraduate Student Research Program (USRP), Pasadena, California, August 2011. 




\end{rSection}


%----------------------------------------------------------------------------------------
%	EXAMPLE SECTION
%----------------------------------------------------------------------------------------

%\begin{rSection}{Section Name}

%Section content\ldots

%\end{rSection}

%----------------------------------------------------------------------------------------

\end{document}
